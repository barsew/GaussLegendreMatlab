\documentclass[a4paper,12pt]{article}
\usepackage{amsfonts}
\usepackage{amssymb}
\usepackage{latexsym}
\usepackage{amsmath}
\usepackage{amsthm}
\usepackage{graphicx}
\usepackage{indentfirst}
\usepackage[polish]{babel}
\usepackage[T1]{fontenc}
\usepackage[utf8]{inputenc} % tu mo�e by� konieczne zast�pienie "cp1250" przez np. "utf8"
\usepackage{setspace}
\usepackage{array}
\usepackage{multirow}
\usepackage{geometry}
\geometry{hdivide={2cm,*,2cm}}
\geometry{vdivide={2cm,*,2cm}}
\usepackage{titlesec}
\titlespacing{\section}{0ex}{1ex}{1ex} % zmniejszenie odst�p�w przed i po tytule rozdzia�u...
\titleformat*{\section}{\sf\large\bfseries} % i zmiana kroju czcionki
\titlespacing{\subsection}{0ex}{0.75ex}{0.75ex} % % j/w dla tytu��w podrozdzia��w
\titleformat*{\subsection}{\sf\bfseries}

% Zmniejszenie odst�p�w przed i za wzorami wystawionymi
\AtBeginDocument{
\addtolength{\abovedisplayskip}{-1ex}
\addtolength{\abovedisplayshortskip}{-1ex}
\addtolength{\belowdisplayskip}{-1ex}
\addtolength{\belowdisplayshortskip}{-1ex}
}
% Kilka przydatnych definicji
\newcolumntype{C}[1]{>{\centering\arraybackslash}m{#1}}
\newcommand{\razy}{\hspace{-0.5ex}\times\hspace{-0.5ex}} % mo�e si� przyda�


\begin{document}

\def\tablename{Tabela} % bez tej linii nazw� tabeli by�aby "Tablica"


\noindent
\textbf{Bartosz Seweryn, 320733, grupa 5, projekt 2, zadanie 14}


\section*{Wstęp}
Obliczanie wartości całki na przedziale $[-1, 1]$ z wielomianu $p$, gdzie $p(x) = a_0T_0(x) + a_1T_1(x) + ... + a_nT_n(x)$ jest wielomianem danym w bazie złożonej z wielomianów Czebyszewa ($T_k$ to $k$-ty wielomian w bazie wielomianów Czebyszewa), $n$-punktową kwadraturą Gaussa-Legendre'a. Z przeprowadznonych eksperymentów numerycznych wynika, że $n$-punkowa kwadratura Gaussa-Legendre'a dokładnie oblicza wartość całki dla wielomianów stopnia mniejszego niż $2n$, natomiast dla wielomianów stopnia większego lub równego $2n$ błędy przybliżeń znacząco rosną. Okazuje się, że zdecydwoanie lepszą metodą jest obliczanie wartości całki wykorzystując własność wielomianów z bazy Czebyszewa, która zostanie przytoczona i udowodniona w dalszej części raportu.


\section*{Opis metody obliczania całek kwadraturą Gaussa-Legendre'a}
Niech 
\[
I(p) = \int_{-1}^{1}p(x)\,dx,
\]
gdzie $p(x) = a_0T_0(x) + a_1T_1(x) + ... + a_nT_n(x)$ jest wielomianem danym w bazie złożonej z wielomianów Czebyszewa. Wartość powyższej całki będziemy przybliżać za pomocą $n$-punktowej kwadratury Gaussa-Legendre'a postaci
\[
S(p) = \sum_{k = 1}^{n}A_kp(x_k),
\]
gdzie $A_i$ to $i$-ty współczynnik $n$-punktowej kwadratury Gaussa-Legendre'a, a $x_k$ to $k$-ty węzeł tej kwadratury.
\section*{Eksperymenty numeryczne}
Udowodnijmy następujący fakt
\[
\int_{-1}^{1}T_{k}(x)\,dx =
\begin{cases}
\frac{2}{1 - k^2} & \text{, gdy k jest parzyste,}\\
0 & \text{, gdy k jest nieparzyste.}
\end{cases}
\]
Dowód. Dla $x \in [-1, 1]$ $T_k(x) = \cos{(k\arccos{x})}$, rozważmy więc całkę
\begin{equation} \label{eq:1}
\int_{-1}^{1} \cos{(k\arccos{x})}\,dx.
\end{equation}
Niech 
\begin{equation} \label{eq:2}
t = \arccos{x},
\end{equation}
wówczas różniczkując stronami otrzymujemy 

\begin{equation} \label{eq:3}
dt = \frac{-dx}{\sqrt{1 - x^2}}.
\end{equation}
Z (\ref{eq:2}) wynika, że 
\[
\cos{t} = x.
\]
Czyli
\begin{equation} \label{eq:4}
x^2 = \cos^2{t}.
\end{equation}
Zatem po podstawieniu równania (\ref{eq:4}) do równania (\ref{eq:3})
\[
dt = \frac{-dx}{\sqrt{1 - \cos^2(t)}}.
\]
Dalej korzystając z jedynki trygonometrycznej i faktu, iż dla $x \in [-1, 1]$ $t$ musi należeć do przedziału $[0, \pi]$, co w konsekwencji oznacza, że $\sin{t} \geq{0}$, otrzymujemy
\[
dt = \frac{-dx}{\sin{t}},
\]
zatem
\begin{equation} \label{eq:5}
dx = -dt\sin{t}.
\end{equation}
Wykorzystując równanie (\ref{eq:5}), podstawienie (\ref{eq:2}) i odpowiednio zmieniając przedział całkowania dostajemy 
\begin{equation} \label{eq:6}
\int_{\pi}^{0} -\cos{(kt)}\sin{t}\,dt = \int_{0}^{\pi} \cos{(kt)}\sin{t}\,dt.
\end{equation}
Korzystając ze znanych tożsamości trygonometrycznych na sinusa sumy i różnicy kątów otrzymujemy wzór
\begin{equation} \label{eq:7}
\sin{\alpha}\cos{\beta} = \frac{1}{2}(\sin{(\alpha + \beta)} + \sin{(\alpha - \beta)}),
\end{equation}
który zostanie użyty do obliczenia całki (\ref{eq:1}).\\
Po podstawieniu $\alpha = t$, $\beta = kt$ do wzoru (\ref{eq:7}), otrzymujemy 
\[
\int_{0}^{\pi} \cos{(kt)}\sin{t}\,dt = \int_{0}^{\pi} \frac{1}{2}(\sin{(t + kt)} + \sin{(t - kt)})\,dt,
\]
\[
\int_{0}^{\pi} \frac{1}{2}(\sin{(t + kt)} + \sin{(t - kt)})\,dt = \frac{1}{2}(\int_{0}^{\pi} \sin{(t + kt)}\,dt +\int_{0}^{\pi} \sin{(t - kt)}\,dt).
\]
Stąd
\[
\int_{-1}^{1} \cos{(k\arccos{x})}\,dx = \frac{1}{2}\frac{-cos(\pi + k\pi) + cos(0)}{1 + k} + \frac{1}{2}\frac{-cos(\pi - k\pi) + cos(0)}{1 - k},
\]
Ostatecznie
\[
\int_{-1}^{1} \cos{(k\arccos{x})}\,dx = \frac{2 - (\cos{(\pi(1 + k))} + \cos{(\pi(1 - k))})}{2(1 - k^2)}
\]
Zauważmy, że dla $k = 2p$, gdzie $p \in \mathbb{Z}$
\[
\int_{-1}^{1} \cos{(2p\arccos{x})}\,dx = \frac{2 - (\cos{\pi(1 + 2p)} + \cos{\pi(1 - 2p)})}{2(1 - 4p^2)} = \frac{2 - (-2)}{2(1 - 4p^2)} = \frac{2}{1 - k^2}
\]
Natomiast dla $k = 2p + 1$
\[
\int_{-1}^{1} \cos{((2p + 1)\arccos{x})}\,dx = \frac{2 - (\cos{(\pi(2 + 2p))} + \cos{(\pi(-2p)}))}{2(1 - (2p + 1)^2)} = \frac{2 - 2}{2(1 - (2p + 1)^2)} = 0.
\]
\filright{\qed}

Testy numeryczne potwierdziły, że dla wielomianów stopnia mniejszego, niż $2n$, gdzie $n$ jest liczbą punktów kwadratury, metoda obliczania całek kwadraturą Gaussa-Legendre'a jest dokładna.
Niestety jeśli wielomian jest większego stopnia, dokładność znacząco spada. Wyniki testu numerycznego zawarte w poniższej tabeli potwierdzają, że nie można polegać na tej metodzie. Znacznie lepszym rozwiązaniem okazuje się policzenie wartości całki używając udowodnionego powyżej faktu. Wówasz wyniki są dokładne dla wielomianów dowolnego stopnia.
\begin{table}[!h]\vspace*{-2ex}
\caption{\footnotesize Wartość błędu $15$-punktowej kwadratury Gaussa-Legendre'a dla wielomianów o losowych współczynnikach podanych stopni.
}\vspace{-1.5ex}
\begin{center}
\begin{tabular}{|l|l|l|l|l|l|}
\hline
Stopień wielomianu & $30$            & $31$            & $32$           & $33$            & $34$            \\ \hline
Błąd               & $6.18\razy10^0$ & $1.55\razy10^0$ & $2.67\razy10^1$ & $2.89\razy10^0$ & $4.74\razy10^0$ \\ \hline
\end{tabular}
\end{center}
\end{table}

\end{document}
